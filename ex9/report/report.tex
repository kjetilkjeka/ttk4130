\documentclass[11pt]{article}

%\usepackage[english]{babel}
\usepackage[utf8]{inputenc}
\usepackage{amsmath, amssymb, amsthm}
\usepackage{gensymb}
\usepackage{graphicx, float}
\usepackage{pstricks-add}


\author{Kjetil Kjeka}
\title{TTK4130 - Exercise 9}
\date{\today}


%Slik at matriser kan defineres med linjer
\makeatletter
\renewcommand*\env@matrix[1][*\c@MaxMatrixCols c]{%
  \hskip -\arraycolsep
  \let\@ifnextchar\new@ifnextchar
  \array{#1}}
\makeatother

\newcommand{\abs}[1]{|#1|} 


\begin{document}
\maketitle
\section*{Problem 1}
\subsection*{a}
Wish to show that $\mathbf{R}_b^a = 
\begin{bmatrix}
\frac{1}{2} \sqrt{3} & \frac{1}{2} & 0 \\
-\frac{1}{2} & \frac{1}{2} \sqrt{3} & 0 \\
0 & 0 & 1
\end{bmatrix}
$ is a rotation matrix by showing it is in $SO(3) = 
\{\mathbf{R} | \mathbf{R} \in \mathbb{R}^{3\times3}, \mathbf{R}^T \mathbf{R} = \mathbf{I}, \det{\mathbf{R}} = 1 \}$. One can see that $\mathbf{R} \in \mathbb{R}^{3\times3}$ must be true since $R_b^a$ is obviously a $3\times3$ matrix. The second property must be true as well since 
\begin{eqnarray*}
\mathbf{R}^T \mathbf{R} &=& 
\begin{bmatrix}
\frac{1}{2} \sqrt{3} & -\frac{1}{2} & 0 \\
\frac{1}{2} & \frac{1}{2} \sqrt{3} & 0 \\
0 & 0 & 1
\end{bmatrix}
\begin{bmatrix}
\frac{1}{2} \sqrt{3} & \frac{1}{2} & 0 \\
-\frac{1}{2} & \frac{1}{2} \sqrt{3} & 0 \\
0 & 0 & 1
\end{bmatrix} \\
&=& 
\begin{bmatrix}
\frac{3}{4} + \frac{1}{4} + 0 & - \frac{3}{4} + \frac{3}{4} + 0 & 0 + 0 + 0 \\
- \frac{3}{4} + \frac{3}{4} + 0 & \frac{1}{4} + \frac{3}{4} + 0 & 0 + 0 + 0 \\
0 + 0 + 0 & 0 + 0 + 0 & 0 + 0 + 1 \\
\end{bmatrix} \\
&=&
\mathbf{I}
\end{eqnarray*}
The last property is also satisfied since
\[\det{\mathbf{R}} = (\frac{1}{2} \sqrt{3})(\frac{1}{2} \sqrt{3})(1) - (\frac{1}{2})(-\frac{1}{2})(1) = 1 \]
And thus the $R_b^a$ is a rotation matrix.

\subsection*{b}
The rotation matrix for the z-axis is
\[\mathbf{R}_z(\psi) = 
\begin{bmatrix}
\cos{\psi} & -\sin{\psi} & 0 \\
\sin{\psi} & \cos{\psi} & 0 \\
0 & 0 & 1
\end{bmatrix}
\]
Seeing that $\mathbf{R}_z(30\degree) = \mathbf{R}_b^a$, meaning that $\mathbf{R}_b^a$ is a $30\degree$ rotation about the z-axis

\subsection*{c}
$\mathbf{R}_a^b$ is the rotation matrix that rotate $-30\degree$ about the z-axis. Since $\mathbf{x} = \mathbf{R}_a^b \mathbf{R}_b^a \mathbf{x} = \mathbf{R}_b^a \mathbf{R}_a^b \mathbf{x}$. We also know that $\mathbf{R}_a^b = (\mathbf{R}_b^a)^{-1} = (\mathbf{R}_b^a)^{T}$.

\subsection*{d}
\[\mathbf{u}^a = \mathbf{R}_b^a \mathbf{u}^b = \begin{bmatrix}
\frac{1}{2} \sqrt{3} - 1 \\
\frac{1}{2} + \sqrt{3} \\
3
\end{bmatrix}
\]

\[\mathbf{w}^b = \mathbf{R}_a^b \mathbf{w}^a = \begin{bmatrix}
\frac{1}{2} \sqrt{3} - \frac{1}{2} \\
-\frac{1}{2} - \frac{1}{2} \sqrt{3} \\
2
\end{bmatrix}
\]

\subsection*{e}
\subsubsection*{i}
$(u^a)^T w^a = (u^b)^T w^b$
\begin{proof}
\begin{eqnarray*}
(u^a)^T w^a &=& (\mathbf{R}_b^a u^b)^T \mathbf{R}_b^a w^b \\
&=& (u^b)^T (\mathbf{R}_b^a)^T \mathbf{R}_b^a w^b \\
&=& (u^b)^T w^b
\end{eqnarray*}
\end{proof}

\subsection*{ii}

\subsection*{f}
Calculating the rotation matrix from a general euler angle can be done like shown below:
\begin{eqnarray*}
\mathbf{R}_b^a &=& \mathbf{R}_{z,\psi} \mathbf{R}_{y,\theta} \mathbf{R}_{x,\phi} \\
&=&
\begin{bmatrix}
c\psi & -s\psi & 0 \\
s\psi & c\psi & 0 \\
0 & 0 & 1 \\
\end{bmatrix}
\begin{bmatrix}
c\theta & 0 & s\theta \\
0 & 1 & 0 \\
-s\theta & 0 & c\theta
\end{bmatrix}
\begin{bmatrix}
1 & 0 & 0 \\
0 & c\phi & -s\phi \\
0 & s\phi & c\phi
\end{bmatrix} \\
&=& 
\begin{bmatrix}
c\psi & -s\psi & 0 \\
s\psi & c\psi & 0 \\
0 & 0 & 1 \\
\end{bmatrix}
\begin{bmatrix}
c\theta & s\theta s\phi & s\theta c\phi \\
0 & c\phi & - s\phi  \\
- s\theta & c\theta s\phi & c\theta c\phi
\end{bmatrix} \\
&=& \begin{bmatrix}
c\psi c\theta  & c\psi s\theta s\phi - s\psi c\phi & c\psi s\theta c\phi + s\psi s\phi \\
c\psi c\theta & s\psi s\theta s\phi + c\psi c\phi &   c\psi s\theta c\phi - s\psi s\theta \\
- s\theta & c\theta s\phi & c\theta c\phi
\end{bmatrix}
\end{eqnarray*}
\subsection*{g}
\subsubsection*{1}
It's easy to see that 
\[\mathbf{R}_1 = \begin{bmatrix}
-1 & 0 & 0 \\
0 & -1 & 0 \\
0 & 0 & 1
\end{bmatrix}
\]
Since then $\det{\mathbf{R}_1} = 1$ and $\mathbf{R}_1 \mathbf{R}_1 = \mathbf{I}$. And it's obvious a 3 by 3 matrix.

\subsubsection*{2}
It's easy to see that 
\[\mathbf{R}_2 = \begin{bmatrix}
0 & 1 & 0 \\
-1 & 0 & 0 \\
0 & 0 & 1
\end{bmatrix}
\]
Since then $\det{\mathbf{R}_2} = 1$ and $\mathbf{R}_2 \mathbf{R}_2 = \mathbf{I}$. And it's obvious a 3 by 3 matrix.

\subsubsection*{3}
To find $\mathbf{R}_3$ one need to solve the set of equations:
\begin{eqnarray*}
\det{\mathbf{R}_3} = 1 \\
\mathbf{R}_3 \mathbf{R}_3^T = \mathbf{I}
\end{eqnarray*}
Since it's obviously a 3 by 3 matrix.

\section*{Problem 2}
\subsection*{a}
The general expression for $A_i$ is:
\begin{eqnarray*}
\mathbf{A}_i &=& Rot_{z,\theta_i} Trans_{x,d_i} Trans_{x,a_i} Rot_{x,\alpha_i} \\
&=& 
\begin{bmatrix} 
\mathbf{R}_z(\theta_i) & \mathbf{0} \\
\mathbf{0}^T & 1
\end{bmatrix}
\begin{bmatrix}
\mathbf{I} & \begin{bmatrix}0 \\ 0 \\ d_i \end{bmatrix} \\
\mathbf{0} & 1
\end{bmatrix}
\begin{bmatrix}
\mathbf{I} & \begin{bmatrix} a_i \\ 0 \\ 0 \end{bmatrix} \\
\mathbf{0} & 1
\end{bmatrix}
\begin{bmatrix} 
\mathbf{R}_z(\theta_i) & \mathbf{0} \\
\mathbf{0}^T & 1
\end{bmatrix} \\
&=&
\begin{bmatrix} 
\mathbf{R}_z(\theta_i) & \begin{bmatrix}0 \\ 0 \\ d_i \end{bmatrix} \\
\mathbf{0}^T & 1
\end{bmatrix}
\begin{bmatrix} 
\mathbf{R}_x(\alpha_i) & \begin{bmatrix} a_1 \\ 0 \\ 0 \end{bmatrix} \\
\mathbf{0}^T & 1
\end{bmatrix} \\
&=&
\begin{bmatrix} 
\mathbf{R}_z(\theta_i)\mathbf{R}_x(\alpha_i) & \begin{bmatrix}a_i \cos{\theta_i} \\ a_i \sin{\theta_i} \\ d_i \end{bmatrix} \\
\mathbf{0}^T & 1
\end{bmatrix}
\end{eqnarray*}

\subsection*{b}
\subsubsection*{Manipilator A}
\[A_1 = 
\begin{bmatrix}
\mathbf{R}_z(q_1) & \begin{bmatrix} l_1 \cos{q_1} \\ l_1 \sin{q_1} \\ 0 \end{bmatrix} \\
\mathbf{0}^T & 1
\end{bmatrix}
=
\begin{bmatrix}
c q_1 & - s q_1 & 0 & l_1 c q_1 \\
s q_1 & c q_1 & 0 & l_1 s q_1 \\
0 & 0 & 1 & 0 \\
0 & 0 & 0 & 1
\end{bmatrix}
\]
\[A_2 =
\begin{bmatrix}
\mathbf{I} & \begin{bmatrix} q_2 \\ 0 \\ 0 \end{bmatrix} \\
\mathbf{0}^T & 1
\end{bmatrix}
=
\begin{bmatrix}
1 & 0 & 0 & q_2 \\
0 & 1 & 0 & 0 \\
0 & 0 & 1 & 0 \\
0 & 0 & 0 & 1
\end{bmatrix}
\]

\subsubsection*{Manipulator B}
\[A_1 = 
\begin{bmatrix}
\mathbf{R}_z(q_1) & \begin{bmatrix} l_1 \cos{q_1} \\ l_1 \sin{q_1} \\ 0 \end{bmatrix} \\
\mathbf{0}^T & 1
\end{bmatrix}
=
\begin{bmatrix}
c q_1 & - s q_1 & 0 & l_1 c q_1 \\
s q_1 & c q_1 & 0 & l_1 s q_1 \\
0 & 0 & 1 & 0 \\
0 & 0 & 0 & 1
\end{bmatrix}
\]

\[A_2 = 
\begin{bmatrix}
\mathbf{R}_z(q_2) & \begin{bmatrix} l_2 \cos{q_2} \\ l_2 \sin{q_2} \\ 0 \end{bmatrix} \\
\mathbf{0}^T & 1
\end{bmatrix}
=
\begin{bmatrix}
c q_2 & - s q_2 & 0 & l_2 c q_2 \\
s q_2 & c q_2 & 0 & l_2 s q_2 \\
0 & 0 & 1 & 0 \\
0 & 0 & 0 & 1
\end{bmatrix}
\]

\subsection*{c}
\subsubsection*{Manipulator A}
\begin{eqnarray*}
T^0_2 &=& A_1 A_2 = 
\begin{bmatrix}
c q_1 & - s q_1 & 0 & l_1 c q_1 \\
s q_1 & c q_1 & 0 & l_1 s q_1 \\
0 & 0 & 1 & 0 \\
0 & 0 & 0 & 1
\end{bmatrix}
\begin{bmatrix}
1 & 0 & 0 & q_2 \\
0 & 1 & 0 & 0 \\
0 & 0 & 1 & 0 \\
0 & 0 & 0 & 1
\end{bmatrix} \\
&=&
\begin{bmatrix}
c q_1 & - s q_1 & 0  & (q_2 + l_1) c q_1 \\
s q_1 & c q_1   & 0  & (q_2 + l_1) s q_1 \\
0     & 0       & 1  & 0        \\
0     & 0       & 0  & 1
\end{bmatrix}
\end{eqnarray*}

\subsubsection*{Manipulator B}
\begin{eqnarray*}
T^0_2 &=& A_1 A_2 = 
\begin{bmatrix}
c q_1 & - s q_1 & 0 & l_1 c q_1 \\
s q_1 & c q_1 & 0 & l_1 s q_1 \\
0 & 0 & 1 & 0 \\
0 & 0 & 0 & 1
\end{bmatrix}
\begin{bmatrix}
c q_2 & - s q_2 & 0 & l_2 c q_2 \\
s q_2 & c q_2 & 0 & l_2 s q_2 \\
0 & 0 & 1 & 0 \\
0 & 0 & 0 & 1
\end{bmatrix} \\
&=&
\begin{bmatrix}
c (q_1 + q_2) & - s (q_1 + q_2) & 0  & l_2 c (q_1 + q_2) + l_1 c q_1 \\
s (q_1 + q_2) & c (q_1 + q_2)   & 0  & l_2 s (q_1 + q_2) + l_1 s q_1 \\
0     & 0       & 1  & 0        \\
0     & 0       & 0  & 1
\end{bmatrix}
\end{eqnarray*}


\subsection*{d}
\subsubsection*{Manipulator A}
\begin{eqnarray*}
g_0 &=& T^0_2 g^2 = 
\begin{bmatrix}
c q_1 & - s q_1 & 0  & (q_2 + l_1) c q_1 \\
s q_1 & c q_1   & 0  & (q_2 + l_1) s q_1 \\
0     & 0       & 1  & 0        \\
0     & 0       & 0  & 1
\end{bmatrix} 
\begin{bmatrix}
1 \\
1 \\
1 \\
1
\end{bmatrix} \\
&=& 
\begin{bmatrix}
(q_2 + l_1 + 1) c q_1 - s q_1 \\
(q_2 + l_1 + 1) s q_1 + s q_1 \\
1 \\
1
\end{bmatrix}
\end{eqnarray*}

\subsubsection*{Manipulator B}
\begin{eqnarray*}
g_0 &=& T^0_2 g^2 = 
\begin{bmatrix}
c (q_1 + q_2) & - s (q_1 + q_2) & 0  & l_2 c (q_1 + q_2) + l_1 c q_1 \\
s (q_1 + q_2) & c (q_1 + q_2)   & 0  & l_2 s (q_1 + q_2) + l_1 s q_1 \\
0     & 0       & 1  & 0        \\
0     & 0       & 0  & 1
\end{bmatrix}
\begin{bmatrix}
1 \\
1 \\
1 \\
1
\end{bmatrix} \\
&=&
\begin{bmatrix}
(l_2 + 1) c (q_1 + q_2) - s (q_1 + q_2) + l_1 c q_1 \\
(l_2 + 1) s (q_1 + q_2) + c (q_1 + q_2) + l_1 s q_1 \\
1 \\
1
\end{bmatrix}
\end{eqnarray*}


\end{document}
