\documentclass[11pt]{article}

%\usepackage[english]{babel}
\usepackage[utf8]{inputenc}
\usepackage{amsmath, amssymb, amsthm}
\usepackage{gensymb}
\usepackage{graphicx, float}
\usepackage{pstricks-add}


\author{Kjetil Kjeka}
\title{TTK4130 - Exercise 9}
\date{\today}


%Slik at matriser kan defineres med linjer
\makeatletter
\renewcommand*\env@matrix[1][*\c@MaxMatrixCols c]{%
  \hskip -\arraycolsep
  \let\@ifnextchar\new@ifnextchar
  \array{#1}}
\makeatother

\newcommand{\abs}[1]{|#1|} 


\begin{document}
\maketitle
\section*{Problem 1}
\subsection*{a}
Wish to show that $\mathbf{R}_b^a = 
\begin{bmatrix}
\frac{1}{2} \sqrt{3} & \frac{1}{2} & 0 \\
-\frac{1}{2} & \frac{1}{2} \sqrt{3} & 0 \\
0 & 0 & 1
\end{bmatrix}
$ is a rotation matrix by showing it is in $SO(3) = 
\{\mathbf{R} | \mathbf{R} \in \mathbb{R}^{3\times3}, \mathbf{R}^T \mathbf{R} = \mathbf{I}, \det{\mathbf{R}} = 1 \}$. One can see that $\mathbf{R} \in \mathbb{R}^{3\times3}$ must be true since $R_b^a$ is obviously a $3\times3$ matrix. The second property must be true as well since 
\begin{eqnarray*}
\mathbf{R}^T \mathbf{R} &=& 
\begin{bmatrix}
\frac{1}{2} \sqrt{3} & -\frac{1}{2} & 0 \\
\frac{1}{2} & \frac{1}{2} \sqrt{3} & 0 \\
0 & 0 & 1
\end{bmatrix}
\begin{bmatrix}
\frac{1}{2} \sqrt{3} & \frac{1}{2} & 0 \\
-\frac{1}{2} & \frac{1}{2} \sqrt{3} & 0 \\
0 & 0 & 1
\end{bmatrix} \\
&=& 
\begin{bmatrix}
\frac{3}{4} + \frac{1}{4} + 0 & - \frac{3}{4} + \frac{3}{4} + 0 & 0 + 0 + 0 \\
- \frac{3}{4} + \frac{3}{4} + 0 & \frac{1}{4} + \frac{3}{4} + 0 & 0 + 0 + 0 \\
0 + 0 + 0 & 0 + 0 + 0 & 0 + 0 + 1 \\
\end{bmatrix} \\
&=&
\mathbf{I}
\end{eqnarray*}
The last property is also satisfied since
\[\det{\mathbf{R}} = (\frac{1}{2} \sqrt{3})(\frac{1}{2} \sqrt{3})(1) - (\frac{1}{2})(-\frac{1}{2})(1) = 1 \]
And thus the $R_b^a$ is a rotation matrix.

\subsection*{b}
The rotation matrix for the z-axis is
\[\mathbf{R}_z(\psi) = 
\begin{bmatrix}
\cos{\psi} & -\sin{\psi} & 0 \\
\sin{\psi} & \cos{\psi} & 0 \\
0 & 0 & 1
\end{bmatrix}
\]
Seeing that $\mathbf{R}_z(30\degree) = \mathbf{R}_b^a$, meaning that $\mathbf{R}_b^a$ is a $30\degree$ rotation about the z-axis

\subsection*{c}
$\mathbf{R}_a^b$ is the rotation matrix that rotate $-30\degree$ about the z-axis. Since $\mathbf{x} = \mathbf{R}_a^b \mathbf{R}_b^a \mathbf{x} = \mathbf{R}_b^a \mathbf{R}_a^b \mathbf{x}$. We also know that $\mathbf{R}_a^b = (\mathbf{R}_b^a)^{-1} = (\mathbf{R}_b^a)^{T}$.

\subsection*{d}
\[\mathbf{u}^a = \mathbf{R}_b^a \mathbf{u}^b = \begin{bmatrix}
\frac{1}{2} \sqrt{3} - 1 \\
\frac{1}{2} + \sqrt{3} \\
3
\end{bmatrix}
\]

\[\mathbf{w}^b = \mathbf{R}_a^b \mathbf{w}^a = \begin{bmatrix}
\frac{1}{2} \sqrt{3} - \frac{1}{2} \\
-\frac{1}{2} - \frac{1}{2} \sqrt{3} \\
2
\end{bmatrix}
\]

\subsection*{e}
\subsubsection*{i}
$(u^a)^T w^a = (u^b)^T w^b$
\begin{proof}
\begin{eqnarray*}
(u^a)^T w^a &=& (\mathbf{R}_b^a u^b)^T \mathbf{R}_b^a w^b \\
&=& (u^b)^T (\mathbf{R}_b^a)^T \mathbf{R}_b^a w^b \\
&=& (u^b)^T w^b
\end{eqnarray*}
\end{proof}

\subsection*{ii}

\end{document}
